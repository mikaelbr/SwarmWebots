\section{E-puck limitations}

The e-puck has certain limitations when it comes to implementing intelligent swarm behaviours.The e-puck has no way of simulating stigmergy, indirect communication between agents using the environment. If we take example an ant colony. The ants leaves a pheromone trail to the food source and back to the nest. The pheromone trail functions as a shared external memory. The trails are also dynamic and fades when a food source is empty. Direct communication can also be a challenge. The e-puck is equipped with IR emitter and receiver so agents can do direct communications with each other it also have bluetooth adapter, but none of the solutions scale well in a swarm. In the master thesis\cite{master}  they have tested out multiple bluetooth connection to epucks they experienced interference problems and delays. Sound signals could have been used to replicate direct communication in a swarm. The e-puck is equipped  with three sound sensors and  one speaker.

Navigation is also a limitation, has no GPS sensor. Like ants and bees that could navigate by using the sun. What the e-puck do have is an accelerometer that could have been used to track the distance travelled by the e-puck. Like for example ants are claimed to have internal pedometers \cite{ants} that they use to keep track of steps so they know the distance they have travelled. 

The epuck have some physical limitations as well, it have no way of grabbing things that leads to issues when retrieving resources. In the  master thesis\cite{master}  they experienced that e-pucks slipped when in trying to move the artificial food source, it have limited pushing power from the master thesis\cite{master} one e-puck is able to push an object 3/4 of its weight, in comparison of the ant that is known have a dragging capacity of several times its body weight. 