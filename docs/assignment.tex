\section{About the project}

The main task of this project was to create a webots controller based on brook’s architecture for behavior-based systems. The robots’ task was to cooperate to find and transport a box(artificial food) to the one of the world’s edges. The strategy is inspired by ants’ swarm behavior, and should be performed without any form of centralized organisation or communication between the individuals.

The e-puck robot used in this project is equipped with proximity and IR sensors, and left and right wheels. The box’ density is set so that it’ll take more than one e-puck to accomplish the task of retrieving the artificial food.

In addition to programming the controller we had to build our world in webots. This world was supposed to simulate an ant’s world, with the box as artificial food and four surrounding walls acting as the hive.

\section{Deliverables}


\begin{enumerate}
	\item Describe briefly the overall system in your words with diagrams and text
	\item Describe the limitations of e-puck to implement such behavior based intelligent swarm behaviors.
	\item Mention the changes if you made in any of the behavior modules to achieve the box-pushing more effectively and/or efficiently.
	\item All design codes used with the project should be delivered.
	\item A working demo of box-pushing task with eight robots (in simulation).
\end{enumerate}
